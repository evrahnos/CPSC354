\documentclass{article}

\usepackage{amsthm}
\usepackage{amsfonts}
\usepackage{amsmath}
\usepackage{amssymb}
\usepackage{enumitem}
\usepackage{fullpage}
\usepackage[usenames]{color}
\usepackage{hyperref}
  \hypersetup{
    colorlinks = true,
    urlcolor = blue,       % color of external links using \href
    linkcolor= blue,       % color of internal links
    citecolor= blue,       % color of links to bibliography
    filecolor= blue,        % color of file links
    }

\usepackage{listings}

\definecolor{dkgreen}{rgb}{0,0.6,0}
\definecolor{gray}{rgb}{0.5,0.5,0.5}
\definecolor{mauve}{rgb}{0.58,0,0.82}

\lstset{frame=tb,
  language=haskell,
  aboveskip=3mm,
  belowskip=3mm,
  showstringspaces=false,
  columns=flexible,
  basicstyle={\small\ttfamily},
  numbers=none,
  numberstyle=\tiny\color{gray},
  keywordstyle=\color{blue},
  commentstyle=\color{dkgreen},
  stringstyle=\color{mauve},
  breaklines=true,
  breakatwhitespace=true,
  tabsize=3
}

\theoremstyle{theorem}
   \newtheorem{theorem}{Theorem}[section]
   \newtheorem{corollary}[theorem]{Corollary}
   \newtheorem{lemma}[theorem]{Lemma}
   \newtheorem{proposition}[theorem]{Proposition}
\theoremstyle{definition}
   \newtheorem{definition}[theorem]{Definition}
   \newtheorem{example}[theorem]{Example}
\theoremstyle{remark}
  \newtheorem{remark}[theorem]{Remark}


\title{CPSC-354 Report}
\author{Eleas Vrahnos  \\ Chapman University}

\date{\today}

\begin{document}

\maketitle

\begin{abstract}
To be written at a later date.
\end{abstract}

\tableofcontents

\section{Introduction}\label{intro}

To be written at a later date.

\section{Homework}\label{homework}

This section will contain my solutions to the weekly homework assignments.

\subsection{Week 1}

The following is a Python implementation of the Euclidean algorithm:

\begin{lstlisting}[language=Python]
def gcd(a,b):
    while a != b:
        if a > b:
            a = a-b
        else:
            b = b-a
    return a
\end{lstlisting}

\newpage % Temporary page break

\noindent We can test this code by going through the function with a sample input \texttt{gcd(9, 33)}, step by step.

\begin{enumerate}[noitemsep]
  \item \texttt{gcd(9, 33)}
  \begin{itemize}
      \item The function is called, assigning 9 to variable \texttt{a} and 33 to variable \texttt{b}.
  \end{itemize}
  \item \texttt{while a != b:}
  \begin{itemize}
      \item The while loop condition returns True, so the loop starts.
  \end{itemize}
  \item \texttt{else:}
  \begin{itemize}
      \item \texttt{a > b} (9 $>$ 33) returns False, so the else block executes.
  \end{itemize}
  \item \texttt{b = b-a}
  \begin{itemize}
      \item \texttt{b} is now assigned to $33 - 9$, which is $24$.
  \end{itemize}
  \item \texttt{while a != b:}
  \begin{itemize}
      \item The while loop condition returns True, so the loop starts.
  \end{itemize}
  \item \texttt{else:}
  \begin{itemize}
      \item \texttt{a > b} (9 $>$ 24) returns False, so the else block executes.
  \end{itemize}
  \item \texttt{b = b-a}
  \begin{itemize}
      \item \texttt{b} is now assigned to $24 - 9$, which is $15$.
  \end{itemize}
  \item \texttt{while a != b:}
  \begin{itemize}
      \item The while loop condition returns True, so the loop starts.
  \end{itemize}
  \item \texttt{else:}
  \begin{itemize}
      \item \texttt{a > b} (9 $>$ 15) returns False, so the else block executes.
  \end{itemize}
  \item \texttt{b = b-a}
  \begin{itemize}
      \item \texttt{b} is now assigned to $15 - 9$, which is $6$.
  \end{itemize}
  \item \texttt{while a != b:}
  \begin{itemize}
      \item The while loop condition returns True, so the loop starts.
  \end{itemize}
  \item \texttt{if a > b:}
  \begin{itemize}
      \item \texttt{a > b} (9 $>$ 6) returns True, so the first block executes.
  \end{itemize}
  \item \texttt{a = a-b}
  \begin{itemize}
      \item \texttt{a} is now assigned to $9 - 6$, which is $3$.
  \end{itemize}
  \item \texttt{while a != b:}
  \begin{itemize}
      \item The while loop condition returns True, so the loop starts.
  \end{itemize}
  \item \texttt{else:}
  \begin{itemize}
      \item \texttt{a > b} (3 $>$ 6) returns False, so the else block executes.
  \end{itemize}
  \item \texttt{b = b-a}
  \begin{itemize}
      \item \texttt{b} is now assigned to $6 - 3$, which is $3$.
  \end{itemize}
  \item \texttt{while a != b:}
  \begin{itemize}
      \item The while loop condition returns False (\texttt{3 == 3}), so the loop ends.
  \end{itemize}
  \item \texttt{return a}
  \begin{itemize}
      \item \texttt{a} is returned from the function, giving the correct greatest common divisor of \textbf{3}.
  \end{itemize}
\end{enumerate}

\section{Project}

To be written at a later date.

\section{Conclusions}\label{conclusions}

To be written at a later date.

\end{document}
